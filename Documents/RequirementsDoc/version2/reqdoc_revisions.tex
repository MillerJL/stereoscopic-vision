\documentclass[10pt,letterpaper,onecolumn]{article}

\usepackage[top=0.5in, bottom=0.5in, left=0.5in, right=0.5in]{geometry}
\usepackage[utf8]{inputenc}
\usepackage{enumerate}

\title{Senior Capstone Requirements Doc Revisions}
\author{Group 54: John Miller, Samuel Schultz, Erin Sullens}
\date{April 27, 2017}
\begin{document}

\maketitle
\vspace{-0.3in}
\noindent
\rule{\linewidth}{0.4pt}

\noindent
While working on the project, learning more about how VR devices work and the capabilities/limitations of systems, we were forced to make some changes to our original project requirements.
These changes are outlined in this paper.
Each change that falls under a similar scope to other changes are grouped in the same section and each section lists what specific items in the requirements document were altered.
These lists follow the format:
\begin{equation}
  1.x.x \; (1.y.y) \; Section Title: \; Description \; of \; Changes
\end{equation}
Where $1.x.x$ is the section of the original requirements doc, and $1.y.y$ is the section on the new requirements doc

\section{Removal of Mobile Application}
This section will detail the alterations made to the requirements document relating to the Mobile Application Requirement.
\subsection{Rational}
As time passed we realized the development of a mobile application would not be able to be completed for several reasons:
\begin{itemize}
  \item Limitations of mobile hardware with requirements of project
  \item Lack of video processing tools available on mobile devices
  \item Lack of time and labor needed to complete mobile app
\end{itemize}
We discussed these reasons with our client, who was understanding and stated that it was okay for us to focus on the desktop application exclusively and drop the mobile application as a requirement.
\subsection{Items Altered}
\begin{itemize}
  \item 1.2 (1.2) Scope: Altered to remove references to mobile application
  \item 1.3.2 (1.3 Definition for Mobile Application): Removed
  \item 2.1 (2.1) Product Perspective: Removed references to mobile application
  \item 2.5 (2.10) Assumptions and Dependencies: Removed portion discussing mobile application
  \item 3.4.1 (3.4.1) Standards Compliance: Removed mobile application
  \item 3.5.3 (3.5.3) Portability: Altered section discussing mobile application to instead discuss other portability solutions.
\end{itemize}

\section{Alterations to Desktop Application UI}
Changes to the Desktop Application UI were necessary as the project evolved.
Details of the changes are outlined here.
\subsection{Rational}
After working on the application for some time and discussing our clients use cases/needs it became apparent that some of the features of the Desktop application were unnecessary and would potentially even hamper its use. Some of these features include:
\begin{itemize}
  \item Items related to managing videos created by the application
  \begin{itemize}
    \item Creating directories and categories to store videos in, under the scope of the application
  \end{itemize}
\end{itemize}
Due to the size of the videos and the technical skill of the client it makes much more sense to let the user manage their own videos.
\subsection{Items Altered}
\begin{itemize}
  \item 2.1.1 (2.2) User Interfaces: Altered to remove mentions of categorizing and viewing videos and added references to user selecting video output locations on file system
  \item 2.1.5 (2.6) Operations: Removed section discussing backup and recovery
  \item 2.5 (2.10) Assumptions and Dependencies: Changed wording surrounding file input assumptions to be more descriptive.
\end{itemize}

\section{Output File Type Selection}
The original specification for output file type selection was based on assumptions by the group which turned out to be unfounded.
This section discusses the changes made to this specification.
\subsection{Rational}
Before becoming familiar with stereoscopic video display, we assumed that there were many different file formats used by all the different devices we wanted to display on.
After working on the project we realized that virtually all relevant 3D video players use a side-by-side format.
Because of this we decided it would be easier for us and the user to just output the processed video file in the single side-by-side format.
\subsection{Items Altered}
\begin{itemize}
  \item 2.1.1 (2.2) User Interfaces: Removed mentions of selecting output file type
  \item 2.1.5 (2.6) Operations: Removed mentions of selecting output file type and added section discussing saving files to user file system
  \item 2.2 (2.7) Product Functions: Changed "Retrieve converted file for use" to "Save final processed video file to host file system"
\end{itemize}

\section{Justification}
We feel that that even without the mobile application the project is complex enough to justify removing it.
The desktop application still provides many different features to work on.
Given that the client is okay with it and there is still enough work to be done we believe these changes should be allowed.
\end{document}
